\documentclass[12pt]{article}

\usepackage{fullpage}
\usepackage{multicol,multirow}
\usepackage{tabularx}
\usepackage{ulem}
\usepackage[utf8]{inputenc}
\usepackage[russian]{babel}
\usepackage{amsmath}
\usepackage{amssymb}
\usepackage{graphicx}
\usepackage{hyperref}

\newcommand{\quotes}[1]{"#1"}


\usepackage{titlesec}

\titleformat{\section}
  {\normalfont\Large\bfseries}{\thesection.}{0.3em}{}

\titleformat{\subsection}
  {\normalfont\large\bfseries}{\thesubsection.}{0.3em}{}

\titlespacing{\section}{0pt}{*2}{*2}
\titlespacing{\subsection}{0pt}{*1}{*1}
\titlespacing{\subsubsection}{0pt}{*0}{*0}
\usepackage{listings}
\lstloadlanguages{Lisp}
\lstset{extendedchars=false,
	breaklines=true,
	breakatwhitespace=true,
	keepspaces = true,
	tabsize=2
}
\begin{document}


\section*{Отчет по лабораторной работе №\,4
по курсу \guillemotleft  Функциональное программирование\guillemotright}
\begin{flushright}
Студент группы 8О-307 МАИ \textit{Спиридонов Кирилл}, \textnumero 18 по списку \\
\makebox[7cm]{Контакты: {\tt vo-ro@list.ru} \hfill} \\
\makebox[7cm]{Работа выполнена: 30.04.22 \hfill} \\
\ \\
Преподаватель: Иванов Дмитрий Анатольевич, доц. каф. 806 \\
\makebox[7cm]{Отчет сдан: \hfill} \\
\makebox[7cm]{Итоговая оценка: \hfill} \\
\makebox[7cm]{Подпись преподавателя: \hfill} \\

\end{flushright}

\section{Тема работы}
{\large Знаки и строки. \par}

\section{Цель работы}
{\large Освоить работу с типами данных CHARACTER, STRING. Научиться применять их на практике.\par}

\section{Задание (вариант №4.36)}
{\large 
Запрограммировать на языке Коммон Лисп функцию, принимающую один аргумент - текст.

Если в тексте нет малых букв, функция должна вернуть этот текст без изменения. 
В противном случае функция должна вернуть копию текста, в котором после всех слов, 
содержащих хотя бы одну малую букву, вставлен знак пунктуации , (запятая).

Функция должна работать как для малых латинских, так и малых русских букв.
}

\section{Оборудование студента}
{\large Процессор Intel(R) Core(TM) i5-8250U CPU @ 1.60GHz, память: 8192Gb, разрядность системы: 64.}

\section{Программное обеспечение}
{\large ОС Ubuntu 20.04 LTS, среда LispWorks Personal Edition 7.1.2}

\section{Идея, метод, алгоритм}
{\large 
Идея алгоритма простая. Создаем новый список(текст), изначально пустой, туда мы будем записывать 
предложения измененные или нет. Сначала пробегаемся по тексту, беря каждый элемент 
списка(предложение). Создаем новое предложение, в которое мы будем копировать слова и другие
символы. Далее пробегаемся по оригинальному предложению, беря каждый символ и записывая 
в локальную переменную word, и если встретится пробел, табуляция или перевод
строки, то проверяем наше слово. Проверяем мы наше слово на то, встретилась ли в
нем lower-case буква, если встретилась, то добавляем к слову запятую, иначе оставляем
слово таким, какое оно есть. Добавляем слово в новое предложение. Так же, когда мы
бежим по оригинальному предложению, мы добавляем все символы не являющиеся словами.
Измененное новое предложение добавляем в новый текст.}

\section{Сценарий выполнения работы}

\section{Распечатка программы и её результаты}

\subsection{Исходный код}
\lstinputlisting{./lab4.lisp}

\subsection{Результаты работы}
% \lstinputlisting{./log2.lisp}

{\large 
(change-text (list \quotes{Ночевала ТУЧКА ЗоЛоТаЯ \\
На груди утеса-великана;} \quotes{А ааа ААА аа     а  })) \\
(\quotes{Ночевала, ТУЧКА ЗоЛоТаЯ, \\
На, груди, утеса-великана;,} \quotes{А ааа, ААА аа,     а,  }) \\
\\
(change-text (list \quotes{Cat  CAT } \quotes{масло ХЛЕБ РУЧКа    СТОЛ})) \\
(\quotes{Cat,  CAT } \quotes{масло, ХЛЕБ РУЧКа,    СТОЛ}) \\
}

\section{Дневник отладки}
\begin{tabular}{|c|c|c|c|}
\hline
Дата & Событие & Действие по исправлению & Примечание \\
\hline
\end{tabular}

\section{Замечания автора по существу работы}
{\large 
	При выполнение столкнулся с такой проблемой: \\
	Если попытаться выполнить следующее выражение: \\
	(concatenate 'string \quotes{абв} \quotes{где} nil \quotes{ёжз}) \\
	То получалась ошибка: \\
	Error: \#\symbol{92}а (of type CHARACTER) is not of type BASE-CHAR. \\
	Чтобы её решить пришлось обратиться к интернету. Решение было найдено на 
	\href{https://stackoverflow.com/questions/57536507/how-to-correctly-detect-file-encodings-with-lispworks}{сайте}. 
	Конкретно помогла конструкция: (lw:set-default-character-element-type 'cl:character). \\
	Так же при попытке определить следующую функцию: \\ 
	(defun whitespace-char-p (char) \\
  (member char '(\#\symbol{92}Space \#\symbol{92}Tab \#\symbol{92}Newline))) \\
  Возникала такая ошибка: Error: Redefining function WHITESPACE-CHAR-P visible from package LISPWORKS { *handle-warn-on-redefinition* is :ERROR }. Т.е. эта функция была уже встроена в мою версию CommonLisp.
}

\section{Выводы}
{\large 
В ходе выполнения лабораторной работы я познакомился с типами данных CHARACTER, STRING. 
Узнал как с ними удобно работать. Воспользовался знаниями из предыдущих лабораторных работ,
используя списки и циклы.
\end{document}
