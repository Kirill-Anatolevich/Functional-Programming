\documentclass[12pt]{article}

\usepackage{fullpage}
\usepackage{multicol,multirow}
\usepackage{tabularx}
\usepackage{ulem}
\usepackage[utf8]{inputenc}
\usepackage[russian]{babel}
\usepackage{amsmath}
\usepackage{amssymb}


\usepackage{titlesec}

\titleformat{\section}
  {\normalfont\Large\bfseries}{\thesection.}{0.3em}{}

\titleformat{\subsection}
  {\normalfont\large\bfseries}{\thesubsection.}{0.3em}{}

\titlespacing{\section}{0pt}{*2}{*2}
\titlespacing{\subsection}{0pt}{*1}{*1}
\titlespacing{\subsubsection}{0pt}{*0}{*0}
\usepackage{listings}
\lstloadlanguages{Lisp}
\lstset{extendedchars=false,
	breaklines=true,
	breakatwhitespace=true,
	keepspaces = true,
	tabsize=2
}
\begin{document}


\section*{Отчет по лабораторной работе №\,2
по курсу \guillemotleft  Функциональное программирование\guillemotright}
\begin{flushright}
Студент группы 8О-307 МАИ \textit{Спиридонов Кирилл}, \textnumero 18 по списку \\
\makebox[7cm]{Контакты: {\tt vo-ro@list.ru} \hfill} \\
\makebox[7cm]{Работа выполнена: 02.04.22 \hfill} \\
\ \\
Преподаватель: Иванов Дмитрий Анатольевич, доц. каф. 806 \\
\makebox[7cm]{Отчет сдан: \hfill} \\
\makebox[7cm]{Итоговая оценка: \hfill} \\
\makebox[7cm]{Подпись преподавателя: \hfill} \\

\end{flushright}

\section{Тема работы}
{\large Простейшие функции работы со списками Коммон Лисп \par}

\section{Цель работы}
{\large Освоить списки в Коммон Лисп. Операции над ними. Применить селекторы.
	Научиться пользоваться map-функционалом. Воспользоваться лямбда-выражениями.\par}

\section{Задание (вариант №2.9)}
{\large 
	Дан список действительных чисел $(x_1,...,x_n), n\ge2.$ \\
	Запрограммируйте рекурсивно на языке Коммон Лисп функцию, вычисляющую выражение вида: \\
	$(x_1 * x_n) + (x_2 * x_{n-1})+...+(x_n * x_1).$ \\

Примеры \\
(sum-product2 '(1 2 3 4 5)) => \\
 (1*5) + (2*4) + (3*3) + (4*2) + (5*1) => \\
 5 + 8 + 9 + 8 + 5 => 35 \\
}

\section{Оборудование студента}
{\large Процессор Intel(R) Core(TM) i5-8250U CPU @ 1.60GHz, память: 8192Gb, разрядность системы: 64.}

\section{Программное обеспечение}
{\large ОС Ubuntu 20.04 LTS, среда LispWorks Personal Edition 7.1.2}

\section{Идея, метод, алгоритм}
{\large 
Идея алгоритма простая. Имеется исходная функция {\tt sum-product2 (l)}, которая 
принимает список и возращает сумму элементов, попарно перемноженных соответствующих
элементов исходного списка с перевернутым этим же списком. Переворачивание списка 
осуществляется при помощи встроенной функции {\tt reverse (l)}. Для поэлементного умножения
воспользуемся лямбда-выражением {\tt(lambda (x1 x2) (* x1 x2))}. В итоге получим список, который состоит из
перемноженных соответствующих элементов исходного списка с перевернутым 
этим же списком. Для него вызываем функцию {\tt sum-of-list (l)}, которая возращает 
сумму всех элементов списка. Полученная сумма и будет ответом на задачу.

}

\section{Сценарий выполнения работы}

\section{Распечатка программы и её результаты}

\subsection{Исходный код}
\lstinputlisting{./lab2.lisp}

\subsection{Результаты работы}
% \lstinputlisting{./log2.lisp}

{\large 
CL-USER 4 > 
(sum-product2 (list 1 2 3 4 5)) \\
35 \\
CL-USER 5 > (sum-product2 (list 3 5 8 9 1)) \\
160 \\
CL-USER 6 > (sum-product2 (list 3 4 4 3)) \\
50 \\
CL-USER 7 > (sum-product2 (list 5 0 1 -2 3)) \\
31 \\
}

\section{Дневник отладки}
\begin{tabular}{|c|c|c|c|}
\hline
Дата & Событие & Действие по исправлению & Примечание \\
\hline
\end{tabular}

\section{Замечания автора по существу работы}
{\large 
Программа работает за O(n).
Можно было бы перемножать попарно не все элементы, а только половину и затем
результат умножать на 2. Но на асимптотику это бы не повлияло.
}

\section{Выводы}
{\large 
В ходе выполнения лабораторной работы я познакомился с типом данных лист. Узнал 
как он устроен (голова + хвост). Применил лямбда-выражение. Понял для чего оно
нужно, а именно, для сокращения кода и лучшей читабельности программы. Приобрёл
навыки обрабатывания списков.

\end{document}
